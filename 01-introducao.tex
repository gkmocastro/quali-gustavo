
\chapter{Introdução}

\section{Contextualização}

Informação de profundidade é uma das representações mais úteis para o entendimento de ambientes físicos \cite{lasinger2019towards} \cite{zhou2019does}. São também uma parte importante da caracterização de relações geométricas de uma determinada cena. As imagens de profundidades (ou mapas de profundidade) desempenham um papel importante em uma série de aplicações que envolvem visão computacional \cite{eigen2014depth}.  Entre elas, podemos citar: compreensão de cenas \cite{jaritz2018sparse}, veículos autônomos \cite{song2021self}, navegação de robôs \cite{ma2019sparse} navegação de VANTs, \cite{padhy2023monocular} fazendas inteligentes \cite{farkhani2019sparse}, e realidade aumentada \cite{du2020depthlab}. 

% \textit{Simultaneos Localization and Mapping} (SLAM) \cite{hu2012robust}

Os mapas de profundidade representam as distâncias de cada ponto (ou pixel) numa cena física em relação ao eixo do dispositivo de captura. Podem ser representados por imagens em escala de cinza, com as cores dos pixels sendo proporcionais à distância, com cinzas mais claros para objetos mais próximos e tons mais escuros para pontos mais afastados (e vice-versa) \cite{dourado2020multi}.

% \begin{figure}[h]
%     \centering
%     \includegraphics[width=\textwidth]{fig/example_depth.png}
%     \caption{Exemplo de mapa de profundidade do \textit{dataset Nyu Depth V2}. No primeiro quadro, a imagem RGB, no segundo, o mapa de profundidade em escala de cinza, no terceiro uma colorização artificial para o mapa de profundidade.}
%     \label{dmap}
% \end{figure}


Para capturar tais imagens geralmente são empregadas câmeras RGB-D, que podem prover tanto informação de profundidade quanto imagens coloridas da cena. Entre suas tecnologias mais comuns, são encontrados diversos tipos de aquisição que podem ser baseados em visão estereoscópica, que trabalha com múltiplos ângulos de visão, sensores \textit{Time-of-Flight} (ToF) que emprega projeção de lasers infravermelhos (IR) estruturados e técnicas mais precisas como o LiDAR (\textit{Light Detection and Ranging}) \cite{castellano2023performance}.




% \begin{figure}
%     \centering   
%     \includegraphics[width=\textwidth]{fig/depth_problema.png}
%     \caption{Exemplo de imagem RGB com mapa de profundidade apresentando leituras inválidas.}
%     \label{errdepth}
% \end{figure}

%problemas 27, 80, 98

Garantir a correta representação dos mapas em escala de pixel é de considerável importância para as tarefas que dependem de profundidade e que requerem um alto grau de segurança e confiabilidade dos dados, como veículos autônomos ou navegação de drones. A tecnologia LiDAR é a alternativa com implementação mais confiável entre as que foram citadas, no entanto, ressalta-se que nem o LiDAR e nem câmeras RGB-D convencionais produzem mapas completos e densos. No caso do LiDAR, são produzidos mapas esparsos (approx. 95\% de esparsidade) e no caso de câmeras RGB-D ou câmeras ToF são produzidos mapas com partes faltantes em determinadas superfícies ou bordas \cite{hu2012robust}. 




% Neste cenário, tecnologias de aquisição e melhoramento dos dados foram amplamente pesquisadas pela ciência nos últimos anos. Recentemente foram exploradas técnicas que não dependem de sensores de profundidade, ou seja, que inferem a informação de profundidade a partir de uma única imagem RGB capturada a partir de uma câmera comum, essa abordagem é conhecida como \textit{Monocular Depth Estimation} (MDE). No entanto, métodos baseados em características puramente visuais   \cite{szeliski2022computer} \cite{hu2022deep}. 



Considerando as limitações impostas por métodos ativos de aquisição de profundidade, surge a possibilidade de inferir um mapa de profundidade denso e completo de uma cena a partir de uma ou mais imagens RGB, processo conhecido como estimação de profundidade (\textit{Depth Estimation - DE}) \cite{rajapaksha2024deep}. Quando duas imagens de câmeras diferentes são utilizadas para obter-se a informação de profundidade, denomina-se \textit{Stereo Matching (SM)}. No entanto, métodos baseados em imagens \textit{stereo} requerem processos complexos de calibração e alinhamento \cite{dong2022towards}.


O problema da estimação monocular de profundidade (\textit{Monocular Depth Estimation - MDE}) tem por objetivo inferir o mapa de profundidade através de uma única imagem RGB. Esse problema é considerado mal-posto devido à ausência de informação geométrica na projeção da cena 3D para a imagem 2D. No entanto, os avanços nas tecnologias de \textit{Deep Learning - DL} e visão computacional tornaram factível e conveniente o uso de MDE para estimar mapas de profundidade densos e completos \cite{spencer2024third} \cite{rajapaksha2024deep}. 

Ao longo dos anos, houveram diversas pesquisas científicas abordando o tema de estimação monocular de profundidade utilizando toda a miríade de técnicas e metodologias dentro do universo do aprendizado profundo, empregando desde redes neurais convolucionais \cite{kopf2021robust}, estruturas \textit{encoder-decoder} \cite{godard2019digging}, mistura de bases de dados em grande escala em modos diferentes \cite{lasinger2019towards}, transformadores de visão \cite{birkl2023midas}, modelos de difusão \cite{ke2024repurposing}, e treinamento utilizando dados reais pseudo-rotulados em larga escala \cite{yang2024depth}. 

Neste cenário, este trabalho propõe uma análise comparativa entre os diversos modelos de estimação monocular de profundidade relativa baseados em aprendizado profundo através da abordagem quantitativa, utilizando métricas e \textit{benchmarks} presentes na literatura, abordagem qualitativa e através de uma aplicação. 



\section{Motivação e Justificativa} 

Os recentes avanços na área de MDE propiciaram a aquisição de informação de profundidade de maneira mais precisa e rápida, dessa forma, também favorecem indiretamente as aplicações que dependem desse tipo de dado, como reconstrução 3D, navegação e veículos autônomos. Além disso, devido à facilidade de implementação dessas técnicas, também podemos citar melhoramentos em aplicações mais modernas como conteúdo gerado por Inteligência Artificial \cite{yang2024depth}.


Reconstruir estruturas 3D a partir de imagens e informação geométrica prévia é um dos tópicos amplamente investigados pela ciência nos últimos anos \cite{zhao2020monocular}. A técnica \textit{Simultaneous Localization and Mapping} (SLAM) consiste planejar e controlar os movimentos de um robô por meio da construção de um mapa espacial do ambiente ao seu redor e obter a sua localização relativa, relacionando a área da visão computacional e a robótica através reconstrução de ambientes 3D e sensores de imagem \cite{placed2023survey} \cite{stachniss2016simultaneous}. Métodos de SLAM que objetivam a fusão de característica de mapas de profundidade obtidos através de sensores em movimento tiveram um aumento de popularidade em tempos recentes, visto que podem ser empregados para navegação e mapeamento de diversos tipos de dispositivos autônomos, como drones e robôs, além das aplicações em realidade aumentada e computação gráfica \cite{tateno2017cnn}.


Detectores de objetos com imagens tem sido aplicados em diversas áreas, como veículos autônomos e visão robótica, em que os sistemas necessitam estimar a localização de pedestres, veículos ou outros obstáculos. Devido à ausência informação de profundidade em imagens 2D, algumas aplicações exigem que a detecção seja feita no espaço tridimensional. O problema da detecção 3D consiste em estimar os vértices das caixas tridimensionais que contenham determinados objetos \cite{hu2022detection}. Uma das formas de adquirir a informação de profundidade necessária é através de LiDAR, entretanto, segundo \citeonline{wu2022sparse}, métodos de detecção 3D baseados somente em informação de LiDAR sofrem com a esparsidade dos dados, além do alto custo financeiro do equipamento. De acordo com \citeonline{ding2020learning} uma alternativa mais desejável é o uso de câmeras monoculares. Mapas de profundidade podem são empregados de duas formas: Transformando os mapas em representação pseudo-LiDAR, ou em sistemas multi-modais em conjunto com a informação RGB. A performance dos métodos de detecção 3D que utilizam informação de profundidade dependem da qualidade e densidade dos mesmos, portanto, as novas tecnologias de estimação monocular profundidade podem contribuir significativamente para este fim.






Segundo \citeonline{khan2020deep}, é evidente o potencial da estimação monocular para problemas de aplicações que envolvam informação de profundidade. Para que as diversas aplicações apresentadas possam funcionar de forma eficaz, é necessário garantir a correta representação dos mapas que são utilizados como entrada dos sistemas. Dado os recentes avanços nos modelos de MDE, tornou-se possível gerar mapas de profundidade de alta qualidade, com fineza de detalhes e de forma rápida, o que cobre as desvantagens de outros tipos de aquisição. 

Neste cenário, faz-se necessária a avaliação dos diversos modelos e técnicas de estimação monocular de profundidade do estado da arte, analisando tanto a sua capacidade de gerar mapas densos e precisos quanto à sua empregabilidade em aplicações práticas que envolvam informação de profundidade, que é a proposta do presente trabalho.

 
\section{Objetivos}


\subsection{Objetivo Geral}
Este trabalho possui como objetivo geral a análise comparativa de estimadores monoculares de profundidade robustos capazes de produzir informação de profundidade de alta qualidade para imagens sob quaisquer circunstâncias.

\subsection{Objetivos Específicos}

\begin{itemize}
    \item Estudo e escolha dos datasets que contenham imagens apropriadas para teste.
    \item Estudo de modelos de estimação monocular de profundidade relativa do estado da arte.
    \item Análise e escolha entre os modelos estudados para implementação e testes.
    \item Avaliação de desempenho perante métricas utilizadas na literatura para comparação entre os modelos no espaço relativo e métrico.
    \item Implementação de método de pós-processamento para transferência do domínio relativo para métrico baseado em transformação de intensidade.
    \item Avaliação qualitativa dos resultados.
    \item Implementação e avaliação de aplicação com os mapas de profundidade gerados a partir dos estimadores do estado da arte.
    
\end{itemize}

