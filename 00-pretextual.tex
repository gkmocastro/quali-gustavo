	% --- -----------------------------------------------------------------
	% --- Elementos usados na Capa e na Folha de Rosto.
	% --- EXPRESS�ES ENTRE <> DEVER�O SER COMPLETADAS COM A INFORMA��O ESPEC�FICA DO TRABALHO
	% --- E OS S�MBOLOS <> DEVEM SER RETIRADOS 
	% --- -----------------------------------------------------------------
	\cleardoublepage
	\thispagestyle{empty}
	
	\vspace{-60mm}
	  \begin{singlespace}
	    \begin{center}
	      {\large UNIVERSIDADE FEDERAL DO ACRE}
	      \vskip4.0cm{\textbf{\Large Gustavo Moreira Oliveira de Castro}}
	      \vskip5.0cm {\textbf{Análise comparativa de estimadores monoculares de profundidade relativa}}
	    \end{center}
	    \begin{center}
	      \vskip10.0cm{\textbf{RIO BRANCO\\2024}}
	    \end{center}
	  \end{singlespace}
	% --- -----------------------------------------------------------------
	% --- Folha de rosto. (Obrigatorio)
	% --- ----------------------------------------------------------------
	\cleardoublepage
	\thispagestyle{empty}
	
	\vspace{-60mm}
	  \begin{center}
	    {\large UNIVERSIDADE FEDERAL DO ACRE}\\
	    \vspace{3cm}
	    {\large Gustavo Moreira Oliveira de Castro} \\
	    \vspace{3cm}
	    
	    {\large Análise comparativa de estimadores monoculares de profundidade relativa} \\
	    \vspace{1.5cm}
	  \end{center}
	
	\noindent
	  \begin{flushright}
	    \begin{minipage}[t]{8cm}
	     \textcolor{black}{Proposta de dissertação de mestrado submetida ao Programa de Pós-Graduação em Ciência da Computação na Universidade Federal do Acre como requisito parcial para obtenção do título de mestre em Ciência da Computação. Linha de Pesquisa: Sistemas Computacionais Inteligentes}
	    \end{minipage}
	  \end{flushright}
	
	\vskip1.50cm
	  \begin{center}
	    \small Orientador: \\
	    Prof. Dr. {Roger Fredy Larico Chavez}\vskip3.5cm 
	  \end{center}
	  \begin{center}
	    \vspace{4mm}
	    RIO BRANCO \\
	    %\vspace{6mm}
	    2024
	  \end{center}
	
	\thispagestyle{empty}
	% --- -----------------------------------------------------------------
	% --- Termo de aprovacao. (Obrigatorio)
	% --- ----------------------------------------------------------------
	\cleardoublepage
	\thispagestyle{empty}
	
	\vspace{-60mm}
	
	\begin{center}
	  {\large Gustavo Moreira Oliveira de Castro} \\
	  \vspace{7mm}
	
	  {\large Análise comparativa de estimadores monoculares de profundidade relativa} \\
	  \vspace{10mm}
	\end{center}
	
	\noindent
	\begin{flushright}
	  \begin{minipage}[t]{8cm}
	
		\textcolor{black}{Proposta de dissertação de mestrado submetida ao Programa de Pós-Graduação em Ciência da Computação na Universidade Federal do Acre como requisito parcial para obtenção do título de mestre em Ciência da Computação. Linha de Pesquisa: Sistemas Computacionais Inteligentes}.
	
	  \end{minipage}
	\end{flushright}
	
	\vspace{1.0 cm}
	\noindent
	
	Approved in <MONTH> of <YEAR>. \\
	\begin{flushright}
	  \parbox{11cm}
	  {
	    \begin{center}
	      \vspace{3mm}
	      \rule{11cm}{.1mm} \\
	      Prof. Dr. Roger Fredy Larico Chavez (Presidente, Orientador)\\
	      Universidade Federal do Acre
	      \vspace{3mm}
	    \end{center}
     
     \begin{center}
	      \vspace{3mm}
	      \rule{11cm}{.1mm} \\
	      Prof. Dra. Ana Beatriz Alvarez Mamani (Co-orientadora)\\
	      Universidade Federal do Acre
	      \vspace{3mm}
	    \end{center}
     
     \begin{center}
	      \vspace{3mm}
	      \rule{11cm}{.1mm} \\
	      Prof. Dr. ...\\
	      Universidade Federal do Acre
	      \vspace{3mm}
	    \end{center}
	  }
	\end{flushright}
	
	\begin{center}
	  \vspace{4mm}
	  RIO BRANCO \\
	  %\vspace{6mm}
	  2024
	\end{center}
	
	% --- -----------------------------------------------------------------
	% --- Dedicatoria.(Opcional)
	% --- ----------------------------------------------------------------
	\cleardoublepage
	\thispagestyle{empty}
	\vspace*{200mm}
	
	\begin{flushright}
	  {\em
	dfsaas
	    \\
	   - 
	    \\ 
	    
	  }
	\end{flushright}
	\newpage
	
	% --- -----------------------------------------------------------------
	% --- Agradecimentos.(Opcional)
	% --- ----------------------------------------------------------------
	\pretextualchapter{Agradecimentos}
	\hspace{5mm}
	
	...
	
	% --- -----------------------------------------------------------------
	% --- Resumo em portugues.(Obrigatorio)
	% --- ----------------------------------------------------------------
	\begin{resumo}
	
	  \begin{center}{
	    \textbf{Análise comparativa de estimadores monoculares de profundidade relativa}}
	  \end{center}
	
	\begin{spacing}{1.0}
		Informação de profundidade possui grande importância em diversas aplicações que exigem informação geométrica, como veículos autônomos, navegação robótica, realidade aumentada e geração de conteúdo por inteligência artificial. Mapas de profundidade podem ser adquiridos com sensores ou através de estimação por métodos passivos. O desenvolvimento de técnicas de aprendizado profundo propiciou a viabilidade da tarefa de estimação monocular de profundidade, em que é utilizada uma única imagem RGB como entrada de um sistema de rede neural. O presente trabalho se propõe a realizar uma análise comparativa dos estimadores monoculares do estado da arte em termos de métricas presentes na literatura, transferência de domínio de relativo para métrico e o emprego em aplicações práticas. Os resultados preliminares demonstraram o funcionamento do método de transformação de intensidades para transferência de domínio, no entanto, ajustes ainda precisam ser feitos no algoritmo para tratamento de funções geradas que não são monotônicas.
	\end{spacing}
	{\hspace{-8mm} \bf{Palavras-chave}}: Mapa de profundidade; Aprendizado Profundo; Detecção de Objetos 3D; Estimação Monocular de Profundidade.
	
	\end{resumo}
	
	% --- -----------------------------------------------------------------
	% --- Resumo em lingua estrangeira.(Obrigatorio)
	% --- ----------------------------------------------------------------
	\begin{abstract}
	
		Depth information has great importance in various applications that require geometric information, such as autonomous vehicles, robotic navigation, augmented reality, and artificial intelligence content generation. Depth maps can be acquired with sensors or through estimation by passive methods. The development of deep learning techniques has made monocular depth estimation a feasible task, where a single RGB image is used as input to a neural network system. This work aims to perform a comparative analysis of state-of-the-art monocular depth estimators in terms of metrics found in the literature, domain transfer from relative to metric, and application in practical scenarios. Preliminary results have demonstrated the effectiveness of the intensity transform method for domain transfer, however, adjustments still need to be made to the algorithm to address generated functions that are not monotonic.
	
	{\hspace{-8mm} \bf{Keywords}}: Depth Map; Deep Learning; 3D Object Detection; Monocular Depth Estimation.
	
	\end{abstract}
	
	% --- -----------------------------------------------------------------
	% --- Lista de figuras.(Opcional)
	% --- -----------------------------------------------------------------
	%\cleardoublepage
	\listoffigures
	
	% --- -----------------------------------------------------------------
	% --- Lista de tabelas.(Opcional)
	% --- -----------------------------------------------------------------
	\cleardoublepage
	%\label{pag:last_page_introduction}
	\listoftables
	\cleardoublepage
	
	% --- -----------------------------------------------------------------
	% --- Sumario.(Obrigatorio)
	% --- -----------------------------------------------------------------
	\pagestyle{ruledheader}
	\tableofcontents